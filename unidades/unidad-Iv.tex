\chapter{Control Polinomial}

En este capítulo se estudiara la retroalimentación por la salida, 
pero utilizando con control basado en los polinomios $B_d(z)$ y $A_d(z)$.

\begin{figure}
    \begin{tikzpicture}[node distance=2.5cm ,auto, >=latex']    
    \node [input, name=input] {};
    \node [block, right of=input] (T) {$T(q)$};
    \node [sum, right of=T] (sum) {$+$};
    \node [block, right of=sum, node distance=2cm] (R) {$\frac{1}{R(q)}$};
    \node [block, right of=R, node distance=3.5cm] (system) {$\frac{B(q)}{A(q)}$};
    \node [output, name=half, right of=system] {};
    \node [output, name=output, right of= half] {};
    \node [block, below of=R] (S) {$S(q)$};


    \draw [draw, ->] (input) -- node {$u_c(k)$} (T);
    \draw [draw, ->] (T) -- node {} (sum);
    \draw [draw, ->] (sum) -- node {} (R);
    \draw [draw, ->] (R) -- node {$u(k)$} (system);
    \draw [draw, -] (system) -- node {$y(k)$} (half);
    \draw [draw, ->] (half) -- node {} (output);
    \draw [draw, ->] (half) |- node {} (S);
    \draw [draw, ->] (S) -| node {$-$} (sum);
\end{tikzpicture} 
    \caption{Sistema retroalimentado por control polinomial.}
    \label{fig:control-polinomial}
\end{figure}

La ley de control propuesta es:

\begin{equation}
    u(k) = \frac{T(q)}{R(q)}u_c(k) - \frac{S(q)}{R(q)}y(k)
\end{equation}

Donde se debe cumplir que: 

\begin{equation}
    grad(T) \leq grad(R) \land grad(S) \leq grad(R)
\end{equation}

Aplicando el teorema de Mason, logramos:

\begin{equation}
    \label{eq:pol-basica}
   \frac{Y(z)}{U_c(z)} = \frac{T(z)B(z)}{R(z)A(z) + S(z)B(z)} = \frac{B_d(z)}{A_d(z)}
\end{equation}


\section{Definiciones previas}

Definiremos que $B(z)$ puede ser expresado como:

\begin{equation}
    B(z) = B^+(z)B^-(z)
\end{equation}

Donde se cumple que: 

\begin{itemize}
    \item $B^+$: Contiene todos los terminos que deseamos eliminar, estos terminos deben estar dentro del circulo unidad.
    \item $B^-$: Contiene todos los terminos que no queremos o podemos eliminar.
    \item $B_d$: Es posible expresarla como $B^-B_d^\prime$.
    \item $R$: Debe contener todos los terminos de $B$ que deseemos eliminar, por lo tanto: $R=B^+R^\prime$.
\end{itemize}

Remplazando en la ecuación ecuación \ref{eq:pol-basica}:

\begin{equation}
    \frac{TB}{B^+R^\prime A+SB} = \frac{TB^-}{R^\prime A + SB^-} = \frac{A_oB^-B^\prime_d}{A_oA^\prime_d}
\end{equation}

Por lo que se desprenden las ecuaciones: 

\begin{equation}
    \siseq{
        T = A_oB^\prime_d \\ 
        R^\prime A + SB^- = A_oA^\prime_d
    }
\end{equation}

Es posible incluir integradores al polinomio $R$ con la finalidad de eliminar el ruido del sistema.

\begin{equation}
    R^\prime = R^{\prime\prime}(z-1)^l
\end{equation}

\section{Solución de la ecuación}

Para resolver la ecuación polinomial seguiremos el siguiente algoritmo: 

\begin{enumerate}
    \item Verificar condiciones de causalidad del proceso y del modelo:
        \subitem $g(A)\geq g(B)$
        \subitem $g(A_d)-g(B_d) \geq g(A) - g(B)$
    \item Factorizar el polinomio $B$ en los ceros que queremos cancelar y los que no: 
        \subitem $B=B^+B^-$
    \item Verificar que el modelo deseado contenga $B^-$:
        \subitem $B_d=B^-B^\prime_d$
    \item Determinar el grado del polinomio observador: 
        \subitem $g(A_o) \geq 2g(A)-g(A_d) -g(B^+) + l -1$
    \item Construir el model del observador, en genearal $z_i=0$:
        \subitem $A_o=\Pi_i^{g(A_o)} (z-z_i)$
    \item Determinar el grado de los polinomios $R^{\prime\prime}$ y $S$:
        \subitem $g(R^{\prime\prime})=g(A_o)+g(A_d)-g(A)-l$
        \subitem $g(S)=l + g(A) - 1$
    \item Escribir $R^{\prime\prime}$ y $S$ en forma de polinomios con coeficientes 
    indeterminados y resolver la ecuación de diseño: 
        \subitem $A(z-1)^lR^{\prime\prime} + B^-S=A_oA_d$
    \item Escribir la let de control: 
        \subitem $R=B^+(z-1)^lR^{\prime\prime}$
        \subitem $T=A_oB^\prime_d$
    \item Verificar la causalidad del controlador: 
        \subitem $g(R) \geq g(T)$
        \subitem $g(R) \geq g(S)$
\end{enumerate}

\section{Análisis de pertubaciones}

La transferencias que se obtienen son las siguientes: 

\begin{equation}
    H_v(q) = \frac{BR}{AR+BS}
\end{equation}

\begin{equation}
    H_e(q) = \frac{BS}{AR+BS}
\end{equation}

Donde $v$ representa una perturbación a la salida del controlador, 
mientras que $e$ representa una perturbación a la salida del sistema.

Si análizamos la expresión de $H_v$, obtenemos que: 

\begin{equation}
    H_v = \frac{B^+(z-1)^lR^{\prime\prime}}{A_oA_d}
\end{equation}

Por lo tanto los polos de la pertubación estan determinados por el observador y del polinomio denominador deseado.
Esto puede producir que el $g(A_o)+g(A_d)$ sea mayor al del sistema desde $u_c$ a $y$, por lo tanto la eliminación 
de la perturbación puede ser más lenta que la convergencia del sistema. 

Al análizar la expresión de $H_e$, obtenemos un resultado similar: 

\begin{equation}
    H_e = \frac{B^+S}{A_oA_d}
\end{equation}

Aunque el orden del sistema dependera de los ceros en común entre $B^+S$ y $A_oA_d$.

